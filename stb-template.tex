\documentclass{book}
\usepackage{fancyhdr}
\pagestyle{fancy}
\fancyhf{} % clear all header and footer fields
\fancyhead[LE,RO]{\thepage} % page number in the header
\fancyhead[LO]{\nouppercase{\rightmark}} % section name on Odd pages (Left)
\fancyhead[RE]{\nouppercase{\leftmark}} % chapter name on Even pages (Right)
\renewcommand{\headrulewidth}{0.4pt} % header rule width
\renewcommand{\footrulewidth}{0pt} % footer rule (set to 0pt if not needed)
\usepackage[utf8]{inputenc}
\usepackage{amsmath}
\usepackage{fancyhdr}
\usepackage{comment}
\usepackage{ragged2e}
\usepackage{titlesec}




\usepackage{amsmath}
\usepackage{comment}
\usepackage{ragged2e}
\usepackage{titlesec}
\usepackage{fancyhdr}
\usepackage{fix-cm}  % this package allows large \fontsize


%\DeclareUnicodeCharacter{2212}{-}

\clearpage
\title{On the Calculation of Mechanical Action}
\newcommand{\booksubtitle}{CONSIDERATIONS ON THE USE OF ENGINES AND THEIR EVALUATION, 
TO SERVE AS AN INTRODUCTION TO THE SPECIAL STUDY OF MACHINES}
%\newcommand{\booklicense}{Creative Commons Zero 1.0 Universal}

\author{Coriolis}

% Create convenient commands \booktitle and \bookauthor
\makeatletter
\newcommand{\booktitle}{\@title}
\newcommand{\bookauthor}{\@author}
\makeatother



% The following dimensions specify 4.75" X 7.5" content on 6 3/8" by 9 1/4"
% paper. The paper width and height can be tweaked as required and the content
% should size to fit within the margins accordingly.
%
% The (inside) bindingoffset should be larger for books with more pages. Some
% standard recommended sizes are .375in minimum up to 1in for 600+ page books.
% Sizes .75in and .875in are also recommended roughly at 150 and 400 pages.
%\usepackage[bindingoffset=0.625in,
           % left=.5in, right=.5in,
           % top=.8125in, bottom=.9375in,
           % paperwidth=6.375in, paperheight=9.25in]{geometry}
% Here is an alternative geometry for reading on letter size paper:
 \usepackage[margin=.75in, paperwidth=6.5 in, paperheight=9.5in]{geometry}



% Content Starts Here
\begin{document}
\frontmatter

% Title page settings



% ---- Half Title Page ----
% current geometry will be restored after title page


% No page numbers on the Frontispiece page
\thispagestyle{empty}
\newpage
\begin{flushleft}
This work is licensed under a Creative Commons Attribution-ShareAlike 4.0 International License (CC BY-SA 4.0). For more details, visit \\https://creativecommons.org/licenses/by-sa/4.0/.\\
\vspace{80mm}
This book was typeset using \LaTeX{} software.\\
\end{flushleft}


\newpage


\newpage
\thispagestyle{empty}
%\begin{flushleft}
\vspace*{\fill}
\textbf{Foreword}
\setlength{\parindent}{20pt}
\vfill

This translation was done by ChatGPT (OpenAI). The final text has been thoroughly proofread and edited by me to ensure accuracy and coherence. 

I have taken care to make minimal changes to the text to preserve the authenticity of the document. Minor revisions were undertaken to correct unnatural phrasing and to clarify mathematical terms and equations. Any changes made are indicated in the LaTeX code of this document.

The term "living force" was not translated into its modern equivalent, "kinetic energy." I am quite fond of this historical phrase and preferred to keep it intact. I think "living force" is much too beautiful of a phrase to remove from this book. 

Considerable effort has been put into ensuring the accuracy of the equations in this book. Despite these efforts, there may still be errors in the translation. If any are discovered, or if there are suggestions for improvements, please feel free to email me. I welcome and appreciate any feedback.

\vfill



 \hfill Samantha White\\


\vfill


\vspace{\fill}






\backmatter
\addcontentsline{toc}{chapter}{Index}

% =========================
% Coriolis — Original title page
% =========================
\newpage
\thispagestyle{empty}

\begin{center}
{\Huge\bfseries ON THE CALCULATION}\\[2mm]
{\Huge\bfseries OF THE EFFECT}\\[2mm]
{\Huge\bfseries OF MACHINES,}\\[4mm]
{\large\bfseries or}\\[4mm]
{\Large\bfseries CONSIDERATIONS ON THE USE OF MOTORS}\\[2mm]
{\Large\bfseries AND ON THEIR EVALUATION,}\\[4mm]
{\large\bfseries TO SERVE AS AN INTRODUCTION TO THE SPECIAL STUDY OF MACHINES;}\\[10mm]

{\large By Coriolis, Engineer of Bridges and Roads.}\\[18mm]

{\Large PARIS,}\\[2mm]
{\large CARILIAN-GOEURY, BOOKSELLER}\\
{\small OF THE ROYAL CORPS OF BRIDGES AND ROADS AND OF MINES,}\\
{\small QUAI DES AUGUSTINS, No. 41.}\\[6mm]
{\large 1829}
\end{center}



% =========================
% Coriolis — Original p. I
% =========================
\newpage

\begin{center}
{\Huge\bfseries NOTICE}
\end{center}

\setlength{\parindent}{20pt}

I have proposed in this work to present all the general considerations that tend to clarify questions concerning the economy of what is commonly called \textit{mechanical force} or \textit{mechanical power}, and to provide means of easily recognizing what are the advantages and disadvantages of certain arrangements in the construction of a machine. I believe that after reading this memoir one will be in a position to proceed suitably in all researches of calculation and experiment that relate to this subject.

The special treatises that have been published up to the present on machines have not completely developed the theory of the use of motors, which in fact seems more naturally to belong within the teaching of rational mechanics. On the other hand, the works that treat of that science contain almost nothing on this theory. I have tried to fill this gap, and thus to provide a useful complement to the mechanics courses of the École Polytechnique, while at the same time offering an introduction to those of the application schools.

The table of contents will show more particularly which questions I have treated; I think that they will present no difficulties to persons who have some notions of infinitesimal analysis and of mechanics. If, however, one wishes to pass over calculations that one would not follow with sufficient ease, one may do so without inconvenience; a simple reading of all the rest will suffice to give the principal notions on the theory of machines(*).

\vspace{2mm}

\noindent (*) I have indicated in the table, by an asterisk, the numbers of the articles that are not necessary to the understanding of the rest of the work, and that one may pass over if one wishes. Even by reading only the first chapter, the third up to Article 73, and the fourth from Article 130 onward, one will already acquire useful notions on machines.

% Translation note:
% - "Avertissement" rendered as "Notice" (common in French books of this period).
% - "force ou puissance mécanique" rendered as "mechanical force or mechanical power" to preserve the author’s pairing rather than modernizing.

% =========================
% Coriolis — Original p. II
% =========================
\newpage

This Memoir was to be the beginning of a more extensive work which I had undertaken long ago, but which other occupations have obliged me to abandon. I had communicated, in 1819, this first part to several persons, among others to Messrs. Mallet, Bélanger, and Drappier, engineers; to several students of the École Polytechnique; and later to M. Ampère in 1820, and to M. Poncelet in 1824. As I no longer foresee that I shall be able to continue this work; as moreover the continuation that it would have required for the special study of machines will be found in the courses that learned professors are about to publish for the application schools of the public services; and as, in addition, the researches on the small vibrations of bodies, which should complete the theory, are being undertaken by distinguished geometers; I have resolved to publish this first Memoir, imperfect though it is, and to give it, with very slight changes, as I had composed it ten years ago: I have added since only a few calculations on the use of wind and of steam.

In my first work I had arrived at considerations which seemed to me new in certain respects. In fact, to my knowledge, there existed on this subject only the work of Carnot and that of M. Gueniveau (*); but at the same time that I was occupied with this theory, Petit inserted in the \textit{Annales de Physique} a succinct Memoir on the use of the principle of \textit{living forces}; and shortly afterward M. Navier published his useful and learned Notes on Bélidor’s \textit{Hydraulic Architecture}. These publications have deprived me today of any priority on certain ideas where it was natural to arrive at the same conclusions, so that in several points this little work will differ from what has been written on the same subject only by the manner in which those same points are treated. Nevertheless, I have thought that it would not be without usefulness to gather together and to present under another form all the considerations that relate to a theory as important as that of machines (*).

\vspace{3mm}

\noindent (*) I have learned only recently of an article published in 1815 by M. Burdin in No.\ 221 of the \textit{Journal des Mines}, in which this engineer gave very sound considerations on machines that, I believe, had not yet been presented so clearly.

% Source: Du calcul de l'effet des machines_pages II, III.pdf :contentReference[oaicite:0]{index=0}


% =========================
% Coriolis — Original p. III
% =========================
\newpage

I have employed in this Work several new denominations: I designate by the name of \textit{work} the quantity that is rather commonly called \textit{mechanical power}, \textit{quantity of action}, or \textit{dynamic effect} (**), and I propose the name \textit{dynamode} for the unit of this quantity. One will find in Articles (16), (30), and (31) the reasons that have led me to make use of these denominations. I have permitted myself another slight innovation in calling \textit{living force} the product of the weight by the height due to the velocity. This living force is only half of the product of what has until now been designated by that name, that is to say, of the mass by the square of the velocity. If one had experienced, as I have, how much students are embarrassed by poorly chosen denominations, I believe one would not censure this slight change. It is very inconvenient to have a name for the double of a quantity that one encounters at every instant. If formerly the name of \textit{living force} was given to the product of the mass by the square of the velocity, it is because attention was not directed toward \textit{work}, and because it was not the product of the weight by the height due to the velocity that one had most often to designate. All practitioners today understand by \textit{living force} the work that can be produced by the velocity acquired by a body; and certainly, whatever one does, there would always be two usages in practice, one applying to a quantity double of the other, if geometers did not adopt the latter, which is in reality the more convenient for the study of the motion of machines. Besides, when

\vspace{3mm}

\noindent (*) M. Bélanger, engineer of Bridges and Roads, author of a very interesting memoir on the flow of water in canals, who has successfully occupied himself with the theory of machines and its applications, was kind enough to review my manuscript and assist me with his advice; I owe to him several improvements that have introduced greater clarity and order into this Work.

\vspace{2mm}

\noindent (**) The word \textit{work} comes so naturally in the sense in which I employ it that, although it has neither been proposed nor recognized as a technical expression, it has nevertheless been employed incidentally by M. Navier in his Notes on Bélidor, and by M. de Prony in his Memoir on the Experiments of the Gros-Caillou Machine.

% Translation notes:
% - "forces vives" rendered as "living forces" (plural preserved).
% - Coriolis’ redefinition of "force vive" preserved as "living force."
% - "puissance mécanique" rendered as "mechanical power" without modernization.
% =========================
% Coriolis — Original p. IV
% =========================
\newpage

Even if one did not wish to introduce this new denomination into rational Mechanics, might one not still permit oneself to use it in works on machines? If readers are well versed in rational Mechanics, this change will not inconvenience them; on the other hand, it will certainly be advantageous for a far greater number of persons who study machines without pursuing further the study of Mechanics.

Some time ago, certain members of the Academy of Sciences having requested that a choice be made for the unit of work or of mechanical power, I then submitted a note to propose the denominations of which I have just spoken. The celebrated Laplace, who was part of a commission appointed for this purpose, was kind enough to tell me that he did not believe that the Academy ought to take the initiative in choosing names, that it could only sanction usage when it began to become established. According to that illustrious geometer, it belonged to those who concerned themselves with machines to attempt to introduce the terms that they judged most suitable.

In accordance with an opinion of such great weight, it seemed to me that I could not be blamed for proposing and employing denominations that appeared to me clearer and more appropriately chosen in their etymology.

% Source: Du calcul de l'effet des machines_pages IV and first TOC.pdf :contentReference[oaicite:0]{index=0}


% =========================
% Coriolis — Original p. (First TOC page)
% =========================
\newpage

\begin{center}
{\Large\bfseries TABLE OF CONTENTS}
\end{center}

\noindent\textit{Note.} The articles marked with an asterisk are those that relate less directly to the general ideas on the effect of machines, and which one may omit if one wishes.

\vspace{4mm}

\begin{center}
\textbf{CHAPTER ONE}
\end{center}
\vspace{2mm}
\noindent
\textit{Articles}\hfill\textit{Pages}

\vspace{2mm}

\noindent
1. Different points of view under which Machines may be studied.\dotfill 1\\
3. Statements of the physical principles that serve as the bases of Dynamics.\dotfill 3\\
4. On the motion of a point.\dotfill 4\\
6. What is meant by connections in Dynamics.\dotfill 6\\
7. General principle of Dynamics.\dotfill 8\\
8. On the principle of virtual velocities.\dotfill 11\\
13. Equation of living forces.\dotfill 14\\
14. On the quantity that appears in the equation of living forces.
\dotfill 15\\
16. Denomination of dynamical work, or simply of work, proposed for this \\ quantity.\dotfill 16\\
17. What is called living force in this Work.\dotfill 17\\
18. Principle on the transmission of work, and modifications of the statement in different cases.\dotfill 18\\
23. When friction is neglected, the denomination of principle of the transmission of work is even better justified.\dotfill 25\\
24. Origin of the denomination of living force.\dotfill 25\\
25. Effect of machines relative to work.\dotfill 26\\
26. How work serves as a basis for the evaluation of a motor.\dotfill 27\\
28. Analogy between work and the volume of materials.\dotfill 30\\
29. Time is an element that remains outside of work.\dotfill 31\\
31. On the unit of work; denomination of dynamode proposed for this unit.\dotfill 33\\
32. Distinction between work and the horizontal transport of burdens.\dotfill 34

\vspace{6mm}

\begin{center}
\textbf{CHAPTER TWO}
\end{center}

\noindent
33. Calculation of the work due to the gravity of bodies in motion.\dotfill 36\\
34. On the work due to mutual reactions, such as springs, attractions, or \\  repulsions.\dotfill 40\\
35. On elastic or imperfectly elastic reactions or springs.\dotfill 43\\
36. On stiffness.\dotfill 44

% Translation note:
% - Preserved "living forces" in plural for "forces vives."
% - Maintained article numbering exactly as printed.
% - Page numbers rendered with \dotfill for visual fidelity.
% =========================
% Coriolis — Original p. VI
% =========================
\newpage

\begin{center}
\textbf{CHAPTER TWO (continued)}
\end{center}

\noindent
37. Influence of stiffness in the distribution of work during the compression or extension of several reactions or springs placed one after another.\dotfill 45\\
40. Influence of masses on the distribution of work in certain particular cases.\dotfill 48\\
41. On the work produced by the pressure of a gas or of a vapor upon a movable envelope whose motion is not too rapid.\dotfill 49\\
43. Calculation of the work that may be collected by employing in different ways the vapor obtained with a given quantity of heat, such as that furnished by the combustion of one kilogram of coal.\dotfill 52\\
44. On the pressure produced on a solid canal by a fluid moving in it with constant velocity.\dotfill 59\\
45. Approximate formula for the pressure produced by a fluid vein against a fixed plane.\dotfill 62\\
46. Approximate formula for the pressure produced against a plane surface immersed in an indefinite current.\dotfill 66\\
47. On the work produced, by a fluid in motion, on a canal and on a movable plane.\dotfill 67\\
48. On the resistant work produced by friction.\dotfill 70\\
50. Equality between the quantities of work produced by equivalent systems of forces.\dotfill 75\\
51. On the work absorbed by friction in a brake.\dotfill 75\\
52. On the calculation of living forces; decomposition of the living force into two terms, one relative to the common motion, and the other to the relative motion.\dotfill 77\\
54. Simplification of the calculation of living forces in the permanent motion of fluids.\dotfill 80\\
55. Extension of the principle of the transmission of work when one abstracts from a uniform motion that carries along a machine.\dotfill 81\\
*56. Application to motion in a movable canal.\dotfill 85\\
*58. The principle of the transmission of work still holds for relative motion with respect to arcs of constant direction passing through the center of gravity of a system in motion.\dotfill 87

\vspace{6mm}

\begin{center}
\textbf{CHAPTER THREE}
\end{center}

\noindent
59. Considerations on the physical constitution of solid bodies.\dotfill 91\\
61. How the principle of the transmission of work may be extended to machines in their physical nature, and in what cases solid bodies can no longer be considered invariable in form.\dotfill 94\\
62. Remarks on the quantities of work due to the mutual actions of particles.\dotfill 95\\
65. Considerations on linear shock.\dotfill 97\\
66. On the shock of two solid bodies, and particularly of elastic bodies.\dotfill 101\\
67. Considerations on elasticity.\dotfill 103\\
69. What may be called the work at each point of a moving body.\dotfill 106\\
71. Considerations on the losses of work due to friction.\dotfill 107

% Source: Du calcul de l'effet des machines_pages VI, VII.pdf :contentReference[oaicite:0]{index=0}


% =========================
% Coriolis — Original p. VII
% =========================
\newpage

\noindent
72. Inductions that may be drawn from the observation of vibrations. General ideas on machines in their true nature.\dotfill 108\\
73. Approximate calculations for the losses of work in the shock of rotational systems when friction is neglected.\dotfill 111\\
75. Some considerations on cases where friction cannot be neglected in these shocks; what diminishes their influence.\dotfill 118\\
*76. Considerations on the Dynamics of quantities of motion in the shock of bodies, and on the principle of d’Alembert.\dotfill 120\\
*78. On the manner of understanding statements in Dynamics relative to quantities of motion when friction is taken into account.\dotfill 124\\
*79. On the Statics and Dynamics of shocks.\dotfill 126\\
*82. On Carnot’s Theorem.\dotfill 129\\
83. How the principle of the transmission of work extends to the motion of incompressible fluids.\dotfill 132\\
*84. Application of the principle of the transmission of work to the flow of liquids.\dotfill 135\\
85. On the work received by a movable vessel into which a liquid vein enters.\dotfill 136\\
87. On the case where the velocity of the vein is not in the same direction as that of the vessel.\dotfill 140\\
88. The pressure borne by the vessel is deduced from the expression that gives the transmitted work.\dotfill 141\\
89. How the principle of the transmission of work extends to elastic fluids.\dotfill 142\\
*90. Application to the flow of air.\dotfill 145\\
91. On the work necessary to expel, by means of a blowing machine, a certain volume of air through a given orifice.\dotfill 148\\
*92. On the pressure received by a movable plane exposed to a current of air.\dotfill 150\\
*93. On the work received by a movable plane through the action of the wind, taking into account the reduction of pressure that occurs behind.\dotfill 156

\vspace{6mm}

\begin{center}
\textbf{CHAPTER FOUR}
\end{center}

\noindent
94. On the different parts of machines intended to produce continuous effects.\dotfill 158\\
95. Theory of flywheels.\dotfill 159\\
96. On the work of waterfalls.\dotfill 165\\
98. On the unit of measure for the work produced by waterfalls as well as other continuous motors.\dotfill 166\\
100. On bucket wheels, that is to say, those in which the water acts largely by its weight.\dotfill 168\\
102. On paddle wheels, that is to say, those in which the water acts after having acquired almost all the velocity due to the fall.\dotfill 174\\
103. On the case where the paddles or blades are perfectly fitted.\dotfill 175\\
104. On the case where the paddles are not fitted and allow water to escape freely along the sides of the paddles.\dotfill 176\\
105. On wheels in which the blades are curved vertically.\dotfill 182\\
106. On paddle wheels when these are wider than the current.\dotfill 183\\
107. On paddle wheels in an indefinite current.\dotfill 185

% Translation notes:
% - Preserved article numbering and asterisk markers exactly as printed.
% - "forces vives" consistently rendered as "living forces."
% - Maintained dot leaders for visual fidelity to the original Table of Contents.

% =========================
% Coriolis — Original p. VIII
% =========================
\newpage

\noindent
108. The speed appropriate to the maximum work to be collected becomes smaller when one evaluates this work only after deducting the losses from its transmission up to a certain point of the machine.\dotfill 186\\
109. Considerations on the establishment of hydraulic wheels and on the means of making them take, in the different cases, the speed corresponding to the maximum for the work to be collected.\dotfill 189\\
114. On the work produced by men and animals, and on the means of collecting it.\dotfill 196\\
118. On some numerical results concerning this work.\dotfill 201\\
119. On the means of collecting the work of steam.\dotfill 202\\
120. In a steam engine, with or without expansion, there exists, for the work to be collected, a maximum which is obtained by placing the temperature of combustion in a suitable ratio with that of the steam.\dotfill 203\\
121. Means of ensuring that the piston takes the appropriate speed.\dotfill 207\\
122. On some experimental results concerning the quantities of work produced by steam engines.\dotfill 207\\
*124. On windmills; establishment of the formulas for finding the elements that correspond to the maximum work to be collected.\dotfill 210\\
*125. Numerical determination of these elements.\dotfill 217\\
*126. On the maximum relative to angular velocity only.\dotfill 220\\
*127. Comparison between the work received by the sails according to theory and according to the experiments of Coulomb.\dotfill 225\\
*129. How it is possible to make the sails take the speed corresponding to the maximum.\dotfill 229\\
130. General considerations on the employment of motors.\dotfill 250\\
131. On the manner of stating precisely the results that indicate the degree of perfection of a machine intended to collect or transmit work.\dotfill 251\\
132. On the manner of establishing the bases of contracts concerning motors.\dotfill 252\\
133. Usefulness of a mechanism suitable for measuring, at a certain point of a machine, the work transmitted; considerations on this subject.\dotfill 253\\
134. Considerations on motion transfers and on the experiments to be made in order to evaluate them.\dotfill 236\\
135. On the different useful effects; how they may be produced while consuming more or less work; distinction between the losses that relate to these effects and those that do not.\dotfill 258\\
136. On experiments concerning the consumption of work required by the various useful effects.\dotfill 240

\vspace{6mm}

\begin{center}
\textbf{TABLES}
\end{center}

\noindent
1°. On the quantities of work required for various useful effects performed by means of machines, according to the most approximate results known today.\dotfill 244\\
2°. On the quantities of work that may be collected from the different motors.\dotfill 255

\vspace{2mm}

\begin{center}
\textbf{NOTE}
\end{center}

\noindent
On a mechanism suitable for measuring with precision the work transmitted by a machine.\dotfill 265

% Source: Du calcul de l'effet des machines_pages VIII, 1.pdf :contentReference[oaicite:0]{index=0}


% =========================
% Coriolis — Original p. 1
% =========================
\newpage

\begin{center}
{\Large\bfseries ON THE CALCULATION}\\[2mm]
{\Large\bfseries OF THE EFFECT}\\[2mm]
{\Large\bfseries OF MACHINES}
\end{center}

\vspace{4mm}

\begin{center}
{\large\bfseries CHAPTER ONE}
\end{center}

\vspace{4mm}

\noindent
Review of the Preliminaries of Dynamics. — Equation of \textit{living forces}. — Principle derived from it for Machines considered from a rational point of view. — Definition of Work as a quantity. — How Work serves to measure the value of a faculty of motion or of a motor. — On the Unit of work. — Distinction between Work and the horizontal transport of burdens.

\vspace{6mm}

\noindent
(1) Machines may be studied from three different points of view: 1°. by considering the forces that arise in the state of equilibrium, as is done in the lever, the screw, the pulleys, and in all machines intended rather to exert great efforts than to produce motion; this is the domain of Statics. 2°. By considering displacement alone in order to understand the dependencies of motion, as is done in the study of the different modes of transmission of motion and of all mechanisms whose object is to substitute for the dexterity of man; this is the domain of Geometry. 3°. Finally, by considering at once the forces and the motion, as is done in machines intended for manufactures of every kind, where the economy of the motor must be taken into account; this is the do-
\newpage
% ----- Original Page 2 -----

marne of Dynamics. It is under this latter point of view alone that we shall study machines.

To be properly understood, it is necessary to recall briefly a few preliminary notions of Mechanics, and to determine clearly in what sense various terms in use in this science will be employed.

The first idea we have of force arises from the sensation we experience when we make an effort to displace a body or to modify its velocity, whether in magnitude or in direction. Inanimate bodies, producing under certain circumstances effects similar to the force of our limbs, we have extended the conception of force to cases where these agents are substituted for our own effort to produce similar effects. From this it followed that forces were compared, and consequently introduced as quantities into calculation. In this respect, force is somewhat like heat, whose first idea was also that of a sensation; it too has been introduced into calculation through its effects in dilating bodies.

(2) Several geometers have considered two kinds of forces. The first, which is that of which I have just spoken, cannot produce instantaneously a perceptible change either in the magnitude or in the direction of the velocity of the body it solicits; this force may be assimilated to the action of a weight or of a spring; it is the only one we shall consider in this work. The second is that which is supposed capable of instantaneously modifying by a finite amount the velocity of a body, either in magnitude or in direction. Although it is recognized that, strictly speaking, this instantaneous modification cannot be admitted, nevertheless, since with a very great force one produces in a very short time perceptible changes, one has permitted oneself to suppose the existence of forces of the kind we have just indicated, conceiving them as, so to speak, infinitely great, yet comparable among themselves. We shall return later, when speaking of shock, to the use that may be made of this dynamics; we shall see that it is founded more upon a certain metaphysics than upon true physics.

Since the consideration of these instantaneous forces is not necessary for what we have to set forth on the theory of machines, we shall not employ them. In what follows, the word
\newpage
% ----- Original Page 3 -----

force shall therefore apply only to that which is analogous to weights, that is to say, to what in several cases is called pressure, tension, or traction. In this sense, a force can never sensibly change the direction or magnitude of a velocity without causing it to pass through all intermediate states, and without this change requiring an appreciable time.

(3) Let us recall briefly what experience has taught concerning the dependence that exists between force and motion.

All observations on motion concur in establishing the following laws:

1°. A body cannot change its velocity in magnitude or in direction unless it is subjected to a force; this is what is called the law of inertia.

2°. In order to produce upon the same body greater or lesser increases or decreases of velocity in equal times, by acting with constant forces directed in the sense of the velocity or in the directly opposite sense, these forces must be proportional to these increases or decreases of velocity; this is the law of proportionality between forces and variations of velocity.

3°. If a body already possessing a velocity comes to be acted upon by a force, and if its motion is referred to three axes passing through the position it would have occupied if this force had not existed; the motion with respect to these axes, whose origin is movable, will be the same as if the body had had no velocity when the force began to act upon it, and the motion had thus been referred to fixed axes. This is the law of independence between acquired motion and the effect of forces.

4°. To produce upon different bodies the same increase of velocity in a given time, forces of different intensities are required. These different intensities, which give rise in us to the idea of masses (*), are proportional to the weights of the bodies taken in the same place of

{\let\thefootnote\relax\footnotetext{(*) It is often said that mass is the quantity of matter of a body; but upon reflection, one immediately sees that as soon as bodies of different natures are involved, and volume can no longer be taken as a measure of mass, we have no other measure of this quantity than the force capable of producing the same accelerated motion.}}

\newpage
% ----- Original Page 4 -----

earth; this is the law of proportionality between masses and weights
for the same place. Since, in the calculation of machines, one does not
have to consider these weights at different places, one may, by virtue
of this law, take weights as quantities proportional to masses.

(4) It is upon these observational laws that the entire dynamics of a
material point is founded. It is easy to conclude from them, for motion
in a straight line, that if one takes as the unit of force any weight,
for example the kilogram, a body whose weight is $p$, being subjected
to a force $F$, which acts with constant intensity during a time $t$
in the direction of motion, will acquire during that time a velocity
equal to $\dfrac{F}{p} g$, designating by $g$ the increase of velocity
of heavy bodies falling vertically during a time $t$, that is to say,
that which the weight $p$ acting alone would produce on the body in
question. If the force $F$ varies with time, the velocity will be given
by the equation
\[
\frac{dv}{dt} = \frac{F}{p} g.
\]

For motion in a curved line, one likewise concludes from these same
laws that if a force $F$ has for components in the directions of three
fixed axes the variable forces $X, Y, Z$, whose unit is the kilogram,
and if it acts alone upon a free material point whose weight,
expressed in kilograms, is represented by $p$, it will produce a motion
such that at each instant one has the equations
\[
\frac{d^2 x}{dt^2} = \frac{X}{p} g,
\qquad
\frac{d^2 y}{dt^2} = \frac{Y}{p} g,
\qquad
\frac{d^2 z}{dt^2} = \frac{Z}{p} g,
\]
where $x, y, z$ are the coordinates of the material point. Thus, the
position of this point and its velocities along the axes being known
at a certain instant, it will suffice to be able to express at each
instant the components $X, Y, Z$ of the force acting on the free point,
in order to deduce, by integration of these three equations, complete
knowledge of the motion.

By means of various considerations developed in treatises on Mechanics,
one concludes from these equations that if one decomposes the force $F$
into two parts, one acting along the tangent to the curve described,
and the other along the normal to this curve, the component along the
tangent will have for expression
\(
\frac{p}{g} \frac{d^2 s}{dt^2},
\)
designating 
\newpage
% ----- Original Page 5 -----
by $s$ the arc described by the moving point; its normal
component will have for expression
\(
\frac{p}{g} \frac{v^2}{r},
\)
designating by $v$ the velocity and by $r$ the radius of curvature (*).

The ratio $\frac{p}{g}$ is what is called the mass; it is ordinarily
designated, for brevity, by $m$; but it has seemed to us that, in order
to avoid the errors that might arise in applications from the use of
the mass $m$, it was more suitable, in this work, to allow the weights
to appear explicitly, so that the kilogram may everywhere be the unit
of force.


(5) In order to pass to the dynamics of a system of material points
connected together in any manner whatsoever, that is to say, to the
determination of the motion of a machine, one must admit that the
same relations which must exist between forces so that they produce
no motion when acting upon points at rest connected by certain
geometrical conditions, are also those which must exist so that these
forces do not modify in any way the motions that these same points,
connected in the same manner, would already have if they were acted
upon by other forces.

For example, if two forces balance each other in a lever at rest,
when they are in the ratio of their distances from the fixed point,
they will in no way modify the motion which the same points would
have if solicited by other forces.

{\let\thefootnote\relax\footnote{(*)
One easily establishes this decomposition in the following manner.
The principles of differential calculus give
\[
\frac{d^2 x}{dt^2}
=
\frac{d}{dt}\left(\frac{dx}{dt}\right)
=
\frac{d}{dt}\left(\frac{dx}{ds} \frac{ds}{dt}\right)
=
\frac{d^2 x}{ds^2} \left(\frac{ds}{dt}\right)^2
+
\frac{dx}{ds} \frac{d^2 s}{dt^2}.
\]
If one denotes by $\alpha$ the angle which the radius of curvature $r$
of the curve described by the moving point makes with the axis of $x$,
and by $a$ the angle which the tangent to this curve makes with the
same axis, one has
\[
\cos \alpha = \frac{d^2 x}{ds^2} r,
\qquad
\cos a = \frac{dx}{ds},
\]
thus the preceding equation becomes
\[
\frac{d^2 x}{dt^2}
=
\frac{1}{r} \left(\frac{ds}{dt}\right)^2 \cos \alpha
+
\frac{d^2 s}{dt^2} \cos a.
\]
Since one has similar equations for $\dfrac{d^2 y}{dt^2}$ and
$\dfrac{d^2 z}{dt^2}$, one concludes that the force which produces
the motion of the moving point decomposes into two forces
\[
\frac{p}{g} \frac{1}{r}\left(\frac{ds}{dt}\right)^2,
\qquad
\frac{p}{g} \frac{d^2 s}{dt^2},
\]
one tangent and the other normal to the curve described.
}}




\end{document}